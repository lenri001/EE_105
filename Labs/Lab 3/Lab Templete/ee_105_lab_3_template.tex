%----------------------------------------------------------------------------------------------------=
%-----------------------------------------------------------------------------------------------------
% Define Document Class, and include/import the multiple libraries used to create this document 
%-----------------------------------------------------------------------------------------------------
\documentclass[12pt]{article}
\usepackage{amsmath}
\usepackage[sc]{mathpazo}
\usepackage{geometry}
\usepackage{datetime}
\usepackage[myheadings]{fullpage}
\usepackage{fancyhdr}
\usepackage{lastpage}
\usepackage{graphicx, wrapfig, subcaption, setspace, booktabs}
\usepackage[T1]{fontenc}
\usepackage[font=small, labelfont=bf]{caption}
\usepackage{fourier}
\usepackage[protrusion=true, expansion=true]{microtype}
\usepackage[english]{babel}
\usepackage{sectsty}
\usepackage{url, lipsum}
\usepackage{hyperref,bookmark}
\usepackage[T1]{fontenc}
\usepackage{amssymb}
\usepackage{listings}
\usepackage{xcolor}
\usepackage{tabularx}
\usepackage{longtable}
\usepackage{setspace}
\usepackage{float}
\usepackage{bigfoot} % to allow verbatim in footnote
\usepackage[framed,numbered,autolinebreaks,useliterate]{matlab-prettifier}
\usepackage{filecontents}
\usepackage[ruled,linesnumbered]{algorithm2e}

%-----------------------------------------------------------------------------------------------------
% Define the heading for this document
%-----------------------------------------------------------------------------------------------------
% Heading style, calling fancy header library
\pagestyle{fancy}
\fancyhf{}
\setlength\headheight{15pt}
%%%% PUT YOUR NAME HERE FOR YOUR HEADING %%%%
\fancyhead[L]{Luis Fernando Enriquez-Contreras}
\fancyhead[R]{EE 105 Lab 3 Solution}
\fancyfoot[R]{Page \thepage\ of \pageref{LastPage}}

%-----------------------------------------------------------------------------------------------------
% Set coding for in LaTeX for Matlab
%-----------------------------------------------------------------------------------------------------
\lstset{
	style              = Matlab-editor,
	basicstyle         = \mlttfamily,
	escapechar         = ",
	mlshowsectionrules = true
}

% The workflow of the document begins here
\begin{document}
	%-------------------------------------------------------------------------------
	% Title
	%-------------------------------------------------------------------------------
	\begin{titlepage}
		
		\newcommand{\HRule}{\rule{\linewidth}{0.5mm}} % Defines a new command for the horizontal lines, change thickness here
		
		\center % Center everything on the page
		
		%---------------------------------------------------------
		%	HEADING SECTIONS
		%---------------------------------------------------------
		
		\textsc{\LARGE University of California, Riverside}\\[1.5cm] % Name of your university/college
		\textsc{\Large Bourns College of Engineering}\\[0.5cm] % Major heading such as course name
		\textsc{\large Department of Electrical and Computer Engineering}\\[0.5cm] % Minor heading such as course title
		
		%---------------------------------------------------------
		%	TITLE SECTION
		%---------------------------------------------------------
		
		\HRule \\[0.6cm]
		{\Large EE 105 Lab 3 Solution \\ \normalsize First-order systems in Simulink}\\[0.4cm] % Title of your document
		\HRule \\[1.0cm]
		
		%---------------------------------------------------------
		%	AUTHOR SECTION
		%---------------------------------------------------------
		
		%\begin{minipage}{0.4\textwidth}
		\begin{center} \large
			% \emph{Authors:}  
			\medskip
			%%% PUT YOUR NAME HERE %%%
			{\textsc{\textbf{Luis Fernando Enriquez-Contreras} }} 
		\end{center}
		%\end{minipage}
		
		
		%---------------------------------------------------------
		%	DATE SECTION
		%---------------------------------------------------------
		\begin{center}
			%			\selectlanguage{USenglish}
			{\large }
		\end{center}
		% Date, change the \today to a set date if you want to be precise
		
		%---------------------------------------------------------
		%	LOGO SECTION
		%---------------------------------------------------------
		%\vfill
		\newcommand*{\plogo}{\includegraphics{Code/Fig/UC_Riverside_seal.pdf}}
		%		\newcommand*{\plogo}{\includegraphics[width=0.25\textwidth]{UC_Riverside_seal.pdf}}
		
		\plogo\\[1cm] % Include a department/university logo - this will require the graphicx package
		
		%---------------------------------------------------------
		
		\vfill % Fill the rest of the page with whitespace
	\end{titlepage}
	
	\newpage
	
	%-------------------------------------------------------------------------------
	% Table of Contents and Figures
	%-------------------------------------------------------------------------------
	%	\doublespacing
	\tableofcontents
	\pagebreak
	\listoffigures
	%	\listoftables
	\lstlistoflistings  
	\pagebreak
	
	%-------------------------------------------------------------------------------
	% BODY
	%-------------------------------------------------------------------------------
	
	\section{Introduction}
	%%% PUT INTRODUCTION HERE %%%
	
	\section{Pre-Lab}
		%%% Insert Figure Here as a .pdf File %%%
		\begin{figure}[H]
			\centering
			\includegraphics[width=1\linewidth]{"prelab.pdf"}
			\caption{prelab}
			\label{fig:prelab}
		\end{figure}
	
	\section{Time Constant Estimation}
		%%% INSERT CODE HERE AS A .m FILE %%%
		\lstinputlisting[language=Matlab, caption = {\Large Matlab Code for Time Constant Estimation}]{code.m}	
		%%% Insert Figure Here as a .pdf File %%%
		\begin{figure}[H]
			\centering
			\includegraphics[width=1\linewidth]{"plot.pdf"}
			\caption{Matlab Data Plot}
			\label{fig:}
		\end{figure}
		% Answer questions pertaining to the section below
		
	\section{Simulation with Simulink}
		\subsection{Zero Input Response}
			\begin{figure}[H]
				\centering
				\includegraphics[width=1\linewidth]{"plot"} % Does not have to be a .pdf, can be another image file
				\caption{Simulink Diagram for Zero Input Response}
				\label{fig:slx_zero_input_diagram}
			\end{figure}	
			%%% Insert Figure Here as a image File %%%
			\begin{figure}[H]
				\centering
				\includegraphics[width=1\linewidth]{"plot"} 
				\caption{Oscilloscope Output of Zero Input Response}
				\label{fig:slx_zero_input_output}
			\end{figure}
			% Answer questions pertaining to the section below
			
		\subsection{Forced Response: Step input}
			\begin{figure}[H]
				\centering
				\includegraphics[width=1\linewidth]{"plot"} % Does not have to be a .pdf, can be another image file
				\caption{Simulink Diagram for Step Input Response}
				\label{fig:slx_step_input_diagram}
			\end{figure}	
			%%% Insert Figure Here as a image File %%%
			\begin{figure}[H]
				\centering
				\includegraphics[width=1\linewidth]{"plot.pdf"} 
				\caption{Oscilloscope Output of Step Input Response}
				\label{fig:slx_step_input_output}
			\end{figure}
			% Answer questions pertaining to the section below
			
		\subsection{Forced Response: Sinusoidal Input}
			\begin{figure}[H]
				\centering
				\includegraphics[width=1\linewidth]{"plot"} % Does not have to be a .pdf, can be another image file
				\caption{Simulink Diagram for Sinusoidal Input Response}
				\label{fig:slx_sine_input_diagram}
			\end{figure}
		
			%%% Insert Figure Here as a image File %%%		
			\begin{figure}[H]
				\centering
				\includegraphics[width=1\linewidth]{"plot.pdf"} 
				\caption{Oscilloscope Output of Sinusoidal Input Response $\omega$ = 0.00}
				\label{fig:slx_sine_input_output_w_0}
			\end{figure}	
				
			%%% Insert Figure Here as a image File %%%
			\begin{figure}[H]
				\centering
				\includegraphics[width=1\linewidth]{"plot.pdf"} 
				\caption{Oscilloscope Output of Sinusoidal Input Response $\omega$ = 0.10}
				\label{fig:slx_sine_input_output_w_0_10}
			\end{figure}	
		
			%%% Insert Figure Here as a image File %%%
			\begin{figure}[H]
				\centering
				\includegraphics[width=1\linewidth]{"plot.pdf"} 
				\caption{Oscilloscope Output of Sinusoidal Input Response $\omega$ = 0.01}
				\label{fig:slx_sine_input_output_w_0_01}
			\end{figure}
		
			%%% Insert Figure Here as a image File %%%
			\begin{figure}[H]
				\centering
				\includegraphics[width=1\linewidth]{"plot.pdf"} 
				\caption{Oscilloscope Output of Sinusoidal Input Response $\omega$ = 1.00}
				\label{fig:slx_sine_input_output_w_1}
			\end{figure}
			
			%%% Insert Figure Here as a image File %%%
			\begin{figure}[H]
				\centering
				\includegraphics[width=1\linewidth]{"plot.pdf"} 
				\caption{Oscilloscope Output of Sinusoidal Input Response $\omega$ = 10.00}
				\label{fig:slx_sine_input_output_w_10}
			\end{figure}
					
			%%% Insert Figure Here as a image File %%%
			\begin{figure}[H]
				\centering
				\includegraphics[width=1\linewidth]{"plot.pdf"} 
				\caption{Oscilloscope Output of Sinusoidal Input Response $\omega$ = 100.00}
				\label{fig:slx_sine_input_output_w_100}
			\end{figure}
		
			% Answer questions pertaining to the section below
		
	\section{Conclusion}
	%%% PUT CONCLUSION HERE %%

\end{document}