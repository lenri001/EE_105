%----------------------------------------------------------------------------------------------------=
%-----------------------------------------------------------------------------------------------------
% Define Document Class, and include/import the multiple libraries used to create this document 
%-----------------------------------------------------------------------------------------------------
\documentclass[12pt]{article}
\usepackage{amsmath}
\usepackage[sc]{mathpazo}
\usepackage{geometry}
\usepackage{datetime}
\usepackage[myheadings]{fullpage}
\usepackage{fancyhdr}
\usepackage{lastpage}
\usepackage{graphicx, wrapfig, subcaption, setspace, booktabs}
\usepackage[T1]{fontenc}
\usepackage[font=small, labelfont=bf]{caption}
\usepackage{fourier}
\usepackage[protrusion=true, expansion=true]{microtype}
\usepackage[english]{babel}
\usepackage{sectsty}
\usepackage{url, lipsum}
\usepackage{hyperref,bookmark}
\usepackage[T1]{fontenc}
\usepackage{amssymb}
\usepackage{listings}
\usepackage{xcolor}
\usepackage{tabularx}
\usepackage{longtable}
\usepackage{setspace}
\usepackage{float}
\usepackage{bigfoot} % to allow verbatim in footnote
\usepackage[framed,numbered,autolinebreaks,useliterate]{matlab-prettifier}
\usepackage{filecontents}
\usepackage[ruled,linesnumbered]{algorithm2e}

%-----------------------------------------------------------------------------------------------------
% Define the heading for this document
%-----------------------------------------------------------------------------------------------------
% Heading style, calling fancy header library
\pagestyle{fancy}
\fancyhf{}
\setlength\headheight{15pt}
%%%% PUT YOUR NAME HERE FOR YOUR HEADING %%%%
\fancyhead[L]{Luis Fernando Enriquez-Contreras}
\fancyhead[R]{EE 105 Lab 1 Solution}
\fancyfoot[R]{Page \thepage\ of \pageref{LastPage}}

%-----------------------------------------------------------------------------------------------------
% Set coding for in LaTeX for Matlab
%-----------------------------------------------------------------------------------------------------
\lstset{
	style              = Matlab-editor,
	basicstyle         = \mlttfamily,
	escapechar         = ",
	mlshowsectionrules = true,
}

% The workflow of the document begins here
\begin{document}
	%-------------------------------------------------------------------------------
	% Title
	%-------------------------------------------------------------------------------
	\begin{titlepage}
		
		\newcommand{\HRule}{\rule{\linewidth}{0.5mm}} % Defines a new command for the horizontal lines, change thickness here
		
		\center % Center everything on the page
		
		%---------------------------------------------------------
		%	HEADING SECTIONS
		%---------------------------------------------------------
		
		\textsc{\LARGE University of California, Riverside}\\[1.5cm] % Name of your university/college
		\textsc{\Large Bourns College of Engineering}\\[0.5cm] % Major heading such as course name
		\textsc{\large Department of Electrical and Computer Engineering}\\[0.5cm] % Minor heading such as course title
		
		%---------------------------------------------------------
		%	TITLE SECTION
		%---------------------------------------------------------
		
		\HRule \\[0.6cm]
		{\Large EE 105 Lab 1 Solution \\ \normalsize MATLAB as an Engineer’s Problem Solving Tool}\\[0.4cm] % Title of your document
		\HRule \\[1.0cm]
		
		%---------------------------------------------------------
		%	AUTHOR SECTION
		%---------------------------------------------------------
		
		%\begin{minipage}{0.4\textwidth}
		\begin{center} \large
			% \emph{Authors:}  
			\medskip
			%%% PUT YOUR NAME HERE %%%
			{\textsc{\textbf{Luis Fernando Enriquez-Contreras} }} 
		\end{center}
		%\end{minipage}
		
		
		%---------------------------------------------------------
		%	DATE SECTION
		%---------------------------------------------------------
		\begin{center}
%			\selectlanguage{USenglish}
			{\large }
		\end{center}
		% Date, change the \today to a set date if you want to be precise
		
		%---------------------------------------------------------
		%	LOGO SECTION
		%---------------------------------------------------------
		%\vfill
		\newcommand*{\plogo}{\includegraphics{Code/Fig/UC_Riverside_seal.pdf}}
%		\newcommand*{\plogo}{\includegraphics[width=0.25\textwidth]{UC_Riverside_seal.pdf}}
		
		\plogo\\[1cm] % Include a department/university logo - this will require the graphicx package
		
		%---------------------------------------------------------
		
		\vfill % Fill the rest of the page with whitespace
	\end{titlepage}
	
	\newpage
	
	%-------------------------------------------------------------------------------
	% Table of Contents and Figures
	%-------------------------------------------------------------------------------
%	\doublespacing
	\tableofcontents
	\pagebreak
	\listoffigures
%	\listoftables
	\lstlistoflistings  
	\pagebreak
	
	%-------------------------------------------------------------------------------
	% BODY
	%-------------------------------------------------------------------------------
	
	\section{Introduction}
		%%% PUT INTRODUCTION HERE %%%
		
	\section{Matrices and Arrays}
		%%% INSERT CODE HERE AS A .m FILE %%%
		\lstinputlisting[language=Matlab, caption = {\Large Matrix Multiplication}]{code.m}
		C = 171.2275
	\section{Scripts}
		%%% INSERT CODE HERE AS A .m FILE %%%
		\lstinputlisting[language=Matlab, caption = {\Large Matrix Multiplication using a For Loop}]{code.m}
		
	\section{More Advanced Scripts}
		\subsection{$f(x_{i})$ Function}
			%%% INSERT CODE HERE AS A .m FILE %%%
			\lstinputlisting[language=Matlab, caption = {$f(x_{i})$ Function}]{code.m}
		\subsection{Plot $f(x_{i})$}
			%%% INSERT CODE HERE AS A .m FILE %%%
			\lstinputlisting[language=Matlab, caption = {\Large Plot Function for $f(x_{i})$}]{code.m}
			\lstinputlisting[language=Matlab, caption = {\Large Code to run function for $f(x_{i})$}]{code.m}
			%%% Insert Figure Here as a .pdf File %%%
			\begin{figure}[H]
				\centering
				\includegraphics[width=1\linewidth]{plot.pdf}
				\caption{\Large $f(x_{i}) = \frac{cos(x_{i})}{1 + e^{3x_{i}}}$ for $N \ = \ 200$}
				\label{fig:fx200}
			\end{figure}
			
			%%% Insert Figure Here as a .pdf File %%%
			\begin{figure}[H]
				\centering
				\includegraphics[width=1\linewidth]{plot.pdf}
				\caption{\Large $f(x_{i}) = \frac{cos(x_{i})}{1 + e^{3x_{i}}}$ for $N \ = \ 5$}
				\label{fig:fx5}
			\end{figure}
			
			%%% Insert Figure Here as a .pdf File %%%
			\begin{figure}[H]
				\centering
				\includegraphics[width=1\linewidth]{plot.pdf}
				\caption{\Large $f(x_{i}) = \frac{cos(x_{i})}{1 + e^{3x_{i}}}$ for $N \ = \ 10$}
				\label{fig:fx10}
			\end{figure}
			  Increasing the number of points (N+1) within the fixed sized region [0,5] for x decreases
			  the space between points, which enhances the resolution. Figure \ref{fig:fx200} exhibits a high domain resolution, resulting in a smooth and continuous plot devoid of discernible edges. Figures \ref{fig:fx5} and \ref{fig:fx10} conversely, demonstrate the effects of a lower domain resolution, characterized by visually apparent edges and a less refined plot appearance. 
			
	\section {Area under the Curve}
		\subsection{Standard Integration}
			%%% INSERT CODE HERE AS A .m FILE %%%
			\lstinputlisting[language=Matlab, caption = {\Large $\int_{0}^{5} \frac{cos(x)}{1 + e^{3x}} dx$}]{code.m}
			\begin{center}
				\LARGE $\int_{0}^{5} \frac{cos(x)}{1 + e^{3x}} dx \approx 0.201$
			\end{center}
			
		\subsection{Riemann Integral Approximation Equations}
			\subsubsection{Right Rectangular Approximation}
				%%% INSERT CODE HERE AS A .m FILE %%%
				\lstinputlisting[language=Matlab, caption = {\Large $\int_{0}^{5} \frac{cos(x)}{1 + e^{3x}} dx$}]{code.m}
				
				%%% Insert Figure Here as a .pdf File %%%
				\begin{figure}[H]
					\centering
					\includegraphics[width=1\linewidth]{plot.pdf}
					\caption{\Large Right rectangular approximation for $Q_{2}$ for $N = 25$}
					\label{fig:q2barplot25}
				\end{figure}
			
				%%% Insert Figure Here as a .pdf File %%%
				\begin{figure}[H]
					\centering
					\includegraphics[width=1\linewidth]{plot.pdf}
					\caption{\Large Right rectangular approximation for $Q_{2}$ for $N = 250$}
					\label{fig:q2barplot250}
				\end{figure}	
						
			\subsubsection{Tradeoffs}
			
			\subsubsection{Trapazoidal approximation derivation}
				
			\newpage
			\subsubsection{$Q_{1}$ and $Q$ vs $N$}
				%%% INSERT CODE HERE AS A .m FILE %%%
				\lstinputlisting[language=Matlab, caption = {\Large $Q_{1}$ Function}]{code.m}
				%%% INSERT CODE HERE AS A .m FILE %%%
				\lstinputlisting[language=Matlab, caption = {\Large Plot $Q_{1}$ and $Q$ vs $N$}]{code.m}
				
				%%% Insert Figure Here as a .pdf File %%%
				\begin{figure}[H]
					\centering
					\includegraphics[width=1\linewidth]{plot.pdf}
					\caption{\Large $Q_{1}$ and $Q$ vs $N$ and the \% Error for $Q_{1}$}
					\label{fig:q1sumerrorplot}
				\end{figure}	
				 $Q_{1}$ in Figure \ref{fig:q1sumerrorplot} converges to $Q$ at around $N \ \approx \ 75$	
				 \newpage
			\subsubsection{$Q_{3}$ and $Q$ vs $N$}
				%%% INSERT CODE HERE AS A .m FILE %%%
				\lstinputlisting[language=Matlab, caption = {\Large $Q_{3}$ Function}]{code.m}
				%%% INSERT CODE HERE AS A .m FILE %%%
				\lstinputlisting[language=Matlab, caption = {\Large Plot $Q_{3}$ and $Q$ vs $N$}]{code.m}
				
				%%% Insert Figure Here as a .pdf File %%%
				\begin{figure}[H]
					\centering
					\includegraphics[width=1\linewidth]{plot.pdf}
					\caption{\Large $Q_{3}$ and $Q$ vs $N$ and the \% Error for $Q_{2}$}
					\label{fig:q2sumerrorplot}
				\end{figure}
				
				%%% Insert Figure Here as a .pdf File %%%
				\begin{figure}[H]
					\centering
					\includegraphics[width=1\linewidth]{plot.pdf}
					\caption{\Large Error Comparison of $Q_{1}$ and $Q_{3}$}
					\label{fig:q1q3errorplot}
				\end{figure}
				
				 
	\section{Conclusion}
		%%% PUT CONCLUSION HERE %%
	
\end{document}