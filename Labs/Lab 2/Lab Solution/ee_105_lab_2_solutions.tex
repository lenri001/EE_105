%----------------------------------------------------------------------------------------------------=
%-----------------------------------------------------------------------------------------------------
% Define Document Class, and include/import the multiple libraries used to create this document 
%-----------------------------------------------------------------------------------------------------
\documentclass[12pt]{article}
\usepackage{amsmath}
\usepackage[sc]{mathpazo}
\usepackage{geometry}
\usepackage{datetime}
\usepackage[myheadings]{fullpage}
\usepackage{fancyhdr}
\usepackage{lastpage}
\usepackage{graphicx, wrapfig, subcaption, setspace, booktabs}
\usepackage[T1]{fontenc}
\usepackage[font=small, labelfont=bf]{caption}
\usepackage{fourier}
\usepackage[protrusion=true, expansion=true]{microtype}
\usepackage[english]{babel}
\usepackage{sectsty}
\usepackage{url, lipsum}
\usepackage{hyperref,bookmark}
\usepackage[T1]{fontenc}
\usepackage{amssymb}
\usepackage{listings}
\usepackage{xcolor}
\usepackage{tabularx}
\usepackage{longtable}
\usepackage{setspace}
\usepackage{float}
\usepackage{bigfoot} % to allow verbatim in footnote
\usepackage[framed,numbered,autolinebreaks,useliterate]{matlab-prettifier}
\usepackage{filecontents}
\usepackage[ruled,linesnumbered]{algorithm2e}

%-----------------------------------------------------------------------------------------------------
% Define the heading for this document
%-----------------------------------------------------------------------------------------------------
% Heading style, calling fancy header library
\pagestyle{fancy}
\fancyhf{}
\setlength\headheight{15pt}
%%%% PUT YOUR NAME HERE FOR YOUR HEADING %%%%
\fancyhead[L]{Luis Fernando Enriquez-Contreras}
\fancyhead[R]{EE 105 Lab 2 Solution}
\fancyfoot[R]{Page \thepage\ of \pageref{LastPage}}

%-----------------------------------------------------------------------------------------------------
% Set coding for in LaTeX for Matlab
%-----------------------------------------------------------------------------------------------------
\lstset{
	style              = Matlab-editor,
	basicstyle         = \mlttfamily,
	escapechar         = ",
	mlshowsectionrules = true,
	otherkeywords = {!,!=,~,$,*,\&,\%/\%,\%*\%,\%\%,<-,<<-,_,/},%
}

% The workflow of the document begins here
\begin{document}
	%-------------------------------------------------------------------------------
	% Title
	%-------------------------------------------------------------------------------
	\begin{titlepage}
		
		\newcommand{\HRule}{\rule{\linewidth}{0.5mm}} % Defines a new command for the horizontal lines, change thickness here
		
		\center % Center everything on the page
		
		%---------------------------------------------------------
		%	HEADING SECTIONS
		%---------------------------------------------------------
		
		\textsc{\LARGE University of California, Riverside}\\[1.5cm] % Name of your university/college
		\textsc{\Large Bourns College of Engineering}\\[0.5cm] % Major heading such as course name
		\textsc{\large Department of Electrical and Computer Engineering}\\[0.5cm] % Minor heading such as course title
		
		%---------------------------------------------------------
		%	TITLE SECTION
		%---------------------------------------------------------
		
		\HRule \\[0.6cm]
		{\Large EE 105 Lab 2 Solution \\ \normalsize Euler's}\\[0.4cm] % Title of your document
		\HRule \\[1.0cm]
		
		%---------------------------------------------------------
		%	AUTHOR SECTION
		%---------------------------------------------------------
		
		%\begin{minipage}{0.4\textwidth}
		\begin{center} \large
			% \emph{Authors:}  
			\medskip
			%%% PUT YOUR NAME HERE %%%
			{\textsc{\textbf{Luis Fernando Enriquez-Contreras} }} 
		\end{center}
		%\end{minipage}
		
		
		%---------------------------------------------------------
		%	DATE SECTION
		%---------------------------------------------------------
		\begin{center}
			%			\selectlanguage{USenglish}
			{\large }
		\end{center}
		% Date, change the \today to a set date if you want to be precise
		
		%---------------------------------------------------------
		%	LOGO SECTION
		%---------------------------------------------------------
		%\vfill
		\newcommand*{\plogo}{\includegraphics{Code/Fig/UC_Riverside_seal.pdf}}
		%		\newcommand*{\plogo}{\includegraphics[width=0.25\textwidth]{UC_Riverside_seal.pdf}}
		
		\plogo\\[1cm] % Include a department/university logo - this will require the graphicx package
		
		%---------------------------------------------------------
		
		\vfill % Fill the rest of the page with whitespace
	\end{titlepage}
	
	\newpage
	
	%-------------------------------------------------------------------------------
	% Table of Contents and Figures
	%-------------------------------------------------------------------------------
	%	\doublespacing
	\tableofcontents
	\pagebreak
	\listoffigures
	\listoftables
	\lstlistoflistings  
	\pagebreak
	
	%-------------------------------------------------------------------------------
	% BODY
	%-------------------------------------------------------------------------------
	
	\section{Introduction}
	%%% PUT INTRODUCTION HERE %%%
	This laboratory exercise aims to equip you with a firm understanding of the practicalities and potential challenges associated with numerically solving differential equations. Through hands-on simulations using MATLAB, you'll gain valuable experience in modeling and analyzing dynamic systems.
	
	\subsection{Objectives}
	\begin{itemize}
		\item Master the concepts: Gain a comprehensive understanding of the key principles and methods involved in numerical differential equation solvers
		\item Develop simulation skills: Learn how to leverage MATLAB's capabilities to construct simulations of dynamic systems accurately and efficiently
		\item Identify potential pitfalls: Be aware of common issues and limitations that can arise when solving differential equations numerically, enabling you to approach numerical solutions with prudence and insight
		\item Apply your knowledge: Put your newfound skills into practice by constructing and analyzing your own dynamic system simulation in MATLAB
	\end{itemize}
	
	\section{Pre-Lab}
		Given the transfer function:
		
		$$
		H(s)=\frac{36}{s^{2}+3 s+36}
		$$
		
		We can determine the following values:
		
		$$
		\begin{aligned}
			\omega_{n} & =6 \\
			G & =1 \\
			\zeta & =\frac{3}{12} = \frac{1}{4}  \\
			\sigma & =\zeta \omega_{n}= \frac{6}{4} = 1.5\\
			\omega_{d} & =\omega_{n} \sqrt{1-\zeta} \approx 5.2
		\end{aligned}
		$$
		$$
		\begin{aligned}
			T_r=\frac{1.8}{\omega_n} \ &
			T_p=\frac{\pi}{\omega_{d}} \  &
			T_s=\frac{4.6}{\sigma} \ &
			M_p(\zeta) = e^{\frac {- \zeta \pi }{\sqrt{1-\zeta^{2}}}}
		\end{aligned}
		$$
		$$
		\begin{aligned}
			T_r= 0.3 \ &
			T_p \approx 0.60384 \ &
			T_s = 3.066\overline{6} \ &
			M_p(\zeta) \approx 0.4329
		\end{aligned}
		$$
		The steady state response is given by:
		
		$$
		\begin{aligned}
			H(s) & =\frac{36}{s^{2}+3 s+36} \\
			\left.H(s)\right|_{s=j \omega} & =\frac{36}{36-\omega^{2}+3 j \omega} \\
			|H(j \omega)|^{2} & =H(j \omega) H(j \omega)^{*} \\
			& =\left(\frac{36}{36-\omega^{2}+3 j \omega}\right)\left(\frac{36}{36-\omega^{2}-3 j \omega}\right) \\
			|H(j \omega)| & =\frac{36}{\sqrt{\left(\left(36-\omega^{2}\right)^{2}+9 \omega^{2}\right)}}  \approx 1.0002\\
			\angle H(j \omega) & =\left(\frac{36}{36-\omega^{2}+3 j \omega}\right)\left(\frac{36-\omega^{2}-3 j \omega}{36-\omega^{2}-3 j \omega}\right) \\
			& =\frac{36\left(36-\omega^{2}-3 j \omega\right)}{\left(36-\omega^{2}\right)^{2}-9 \omega^{2}} = 0.7856^{\circ}
		\end{aligned}
		$$
		
		The steady state response is given by the following:
		
		$$
		y(t)=1.0002 \sin \left(0.1 t-0.7856^{\circ}\right)
		$$
		
		Given that $x=\left[\begin{array}{ll}y & \dot{y}\end{array}\right]^{T}$
		
		$$
		\begin{aligned}
			\dot{x}_{1} & =x_{2} \\
			\dot{x}_{2} & =\ddot{y}(t) \\
			\frac{Y(s)}{U(s)} & =\frac{36}{s^{2}+3 s+36} \\
			Y(s)\left(s^{2}+3 s+36\right) & =36 U(s) \\
			s^{2} Y(s)+3 s Y(s)+36 Y(s) & =36 U(s) \\
			\mathcal{L}^{-1}\left[s^{2} Y(s)+3 s Y(s)+36 Y(s)\right] & =\mathcal{L}^{-1}[36 U(s)] \\
			\ddot{y}(t)+3 \dot{y}(t)+36 y(t) & =36 u(t) \\
			\ddot{y}(t) & =36 u(t)-3 \dot{y}(t)-36 y(t) \\
			\ddot{y}(t) & =36 u(t)-3 x_{2}-36 x_{1} \\
			\dot{x}_{2} & =36 u(t)-3 x_{2}-36 x_{1} \\
			\dot{x} & =\left[x_{2}; 36 u(t)-3 x_{2}-36 x_{1}\right] \\
			y & =\left[x_{1}\right]
		\end{aligned}
		$$
		
		The system is linear, therefore we can solve for $A, B, C, D$
		
		$$
		\begin{array}{ll}
			A=\left[\begin{array}{cc}
				0 & 1 \\
				-36 & -3
			\end{array}\right] & B=\left[\begin{array}{c}
				0 \\
				36
			\end{array}\right] \\
			C=\left[\begin{array}{ll}
				1 & 0
			\end{array}\right] & D=[0]
		\end{array}
		$$
	\section{Linear System Function}
		%%% INSERT CODE HERE AS A .m FILE %%%
		\lstinputlisting[language=Matlab, caption = {\Large lsim Simulation of the State Space Model with domain of 0-100 seconds}]{Code/lsim_run_100.m}
		This code generates the plot for Figure \ref{fig:lsimrun100s}.
		\newpage
		%%% INSERT CODE HERE AS A .m FILE %%%
		\lstinputlisting[language=Matlab, caption = {\Large lsim Simulation of the State Space Model with domain of 0-4 seconds}]{Code/lsim_run_4.m}
		This code generates the plot for Figure \ref{fig:lsimrun4s}.
		%%% Insert Figure Here as a .pdf File %%%
		\begin{figure}[H]
			\centering
			\includegraphics[width=1\linewidth]{"Code/Fig/lsim_run_100s.pdf"}
			\caption{lsim Simulation of the State Space Model with a domain of 0-100 seconds}
			\label{fig:lsimrun100s}
		\end{figure}
		This is the plot for Listing 1.
		%%% Insert Figure Here as a .pdf File %%%
		\begin{figure}[H]
			\centering
			\includegraphics[width=1\linewidth]{"Code/Fig/lsim_run_4s.pdf"}
			\caption{lsim Simulation of the State Space Model with a domain of 0-4 seconds}
			\label{fig:lsimrun4s}
		\end{figure}
		This is the plot for Listing 2.
		\newpage
	\section{Euler Integration Routine}
		%%% INSERT CODE HERE AS A .m FILE %%%
		\lstinputlisting[language=Matlab, caption = {\Large Run the Euler function for different step sizes $h$}]{Code/run_Euler.m}
		
		%%% INSERT CODE HERE AS A .m FILE %%%
		\lstinputlisting[language=Matlab, caption = {\Large Function to run Euler recursion}]{Code/Euler.m}
		
		%%% INSERT CODE HERE AS A .m FILE %%%
		\lstinputlisting[language=Matlab, caption = {\Large Function for the State Space of the representation of the system}]{Code/f.m}
			
		%%% Insert Figure Here as a .pdf File %%%
		\begin{figure}[H]
			\centering
			\includegraphics[width=1\linewidth]{"Code/Fig/Euler_plot_h_1.pdf"}
			\caption{Euler Plot for $h$ = 1}
			\label{fig:eulerploth1}
		\end{figure}	
		%%% Insert Figure Here as a .pdf File %%%
		\begin{figure}[H]
			\centering
			\includegraphics[width=1\linewidth]{"Code/Fig/Euler_plot_h_0.1.pdf"}
			\caption{Euler Plot for $h$ = 0.1}
			\label{fig:eulerploth01}
		\end{figure}
		%%% Insert Figure Here as a .pdf File %%%
		\begin{figure}[H]
			\centering
			\includegraphics[width=1\linewidth]{"Code/Fig/Euler_plot_h_0.05.pdf"}
			\caption{Euler Plot for $h$ = 0.05}
			\label{fig:eulerploth005}
		\end{figure}
		
		\begin{figure}[H]
			\centering
			\includegraphics[width=1\linewidth]{"Code/Fig/Euler_plot_h_0.01.pdf"}
			\caption{Euler Plot for $h$ = 0.01}
			\label{fig:eulerploth001}
		\end{figure}
		%%% Insert Figure Here as a .pdf File %%%
		\begin{figure}[H]
			\centering
			\includegraphics[width=1\linewidth]{"Code/Fig/Euler_plot_h_0.001.pdf"}
			\caption{Euler Plot for $h$ = 0.001}
			\label{fig:eulerploth0001}
		\end{figure}
	
		The curves in Figures \ref{fig:eulerploth1}, \ref{fig:eulerploth01} do not settle at all. The curve in Figure \ref{fig:eulerploth005} does settle, but does so at around 7-8 seconds. The curves Figures \ref{fig:eulerploth001}, \ref{fig:eulerploth0001} do settle at around 3 seconds which is close to $T_{S}$ calculated. 
		\begin{table}[H]
			\centering
			\caption{Cputimes for Various Euler's Step Sizes}
			\begin{tabular}{|c|r|r|r|r|r|r|r|r|}
				\hline & \multicolumn{2}{|c|}{$h=1.0 \mathrm{~s}$} & \multicolumn{2}{c|}{$h=0.1 \mathrm{~s}$} & \multicolumn{2}{c|}{$h=0.01 \mathrm{~s}$} & \multicolumn{2}{|c|}{$h=0.001 \mathrm{~s}$} \\
				\hline Time, $t$ & $k$ & $x_k$ & $k$ & $x_k$ & $k$ & $x_k$ & $k$ & $x_k$ \\
				\hline 0.0 & 0 & 3 & 0 & 3 & 0 & 3 & 0 & 3 \\
				2.0 & 2 & 3 & 20 & 0.9790 & 200 & 0.289 & 2000 & 0.2508 \\
				4.0 & 4 & 6.5940 & 40 & -3.6305 & 400 & 0.3764 & 4000 & 0.3775 \\
				6.0 & 6 & -3784.1155 & 60 & -14.03887 & 600 & 0.5561 & 6000 & 0.5574 \\
				8.0 & 8 & 3752170.88274 & 80 & -30.2920 & 800 & 0.7109 & 8000 & 0.7116\\
				10.0 & 10 & -12125600.5695 & 100 & -45.5193 & 1000 & 0.8366 & 10000 & 0.8371 \\
				\hline CPUTIME (ms) & & 0.054 & & 0.234 & & 0.554 & & 5.219 \\
				\hline
			\end{tabular}
			\label{tab:euler}
		\end{table}
		After running the sine input with different step-sizes, the computational time and accuracy are shown in Table \ref{tab:euler}. The following conclusions can be made'
		\begin{itemize}
			\item There is a direct relationship between step-size and computational cost. Reducing the step-size by 10x leads to a roughly 10x increase in computation time. This suggests that the chosen numerical method might not be very efficient for small step-sizes, and alternative methods should be considered if high accuracy is required at small scales.
			\item The simulation exhibits stability issues for large step-sizes (h > 0.01). This indicates that the chosen step-size might be too large to capture the dynamics of the system accurately. Stable simulations are crucial for obtaining reliable results, so it's important to address this instability
			\item For smaller step-sizes (h < 0.01), the simulation becomes stable and the accuracy improves as the step-size decreases. This suggests that the chosen numerical method is more accurate for smaller step-sizes, but at the cost of increased computational effort.
			\item There is a trade-off between the accuracy of the simulation and the computational cost. Smaller step-sizes lead to more accurate results but require more computation time. The optimal step-size depends on the specific requirements of the study. If high accuracy is essential, even if it means longer computation times, then smaller step-sizes should be used. However, if computational efficiency is a priority, then a larger step-size might be acceptable if the accuracy remains within acceptable limits.
		\end{itemize}
	\section{ODE23 with Zero Input}
	
		\begin{figure}[H]
			\centering
			\includegraphics[width=1\linewidth]{Code/Fig/ode_no_input}
			\caption{ODE Plot for Zero Input}
			\label{fig:odenoinput}
		\end{figure}
		
	\newpage			
	\section{Varying $\omega$ in 1.0 sin($\omega$t) using ODE23}
		%%% INSERT CODE HERE AS A .m FILE %%%
		\lstinputlisting[language=Matlab, caption = {\Large Function for the State Space of the representation of the system for ODE23}]{Code/f_ode.m}
		
		%%% INSERT CODE HERE AS A .m FILE %%%
		\lstinputlisting[language=Matlab, caption = {\Large Run ODE23 for sin(0.1t)}]{Code/ode_sin_input.m}
		
		\begin{figure}[H]
			\centering
			\includegraphics[width=1\linewidth]{Code/Fig/ode_sin_input_0.1}
			\caption{Ode Plot for $\omega$ = 0.1}
			\label{fig:odesininput0_1}
		\end{figure}
		Magnitude and Phase shift of the system response in Figure \ref{fig:odesininput0_1} are near identical to the input signal.
		
		%%% INSERT CODE HERE AS A .m FILE %%%
		\lstinputlisting[language=Matlab, caption = {\Large Run ODE23 for various sin($\omega$t)}]{Code/ode_sin_input_omega.m}	
		%%% Insert Figure Here as a .pdf File %%%
		\begin{figure}[H]
			\centering
			\includegraphics[width=1\linewidth]{Code/Fig/ode_sin_input_1}
			\caption{Ode Plot for $\omega$ = 1}
			\label{fig:odesininput1}
		\end{figure}
		Magnitude and Phase shift of the system response in Figure \ref{fig:odesininput1} are near identical to the input signal.
		%%% Insert Figure Here as a .pdf File %%%
		\begin{figure}[H]
			\centering
			\includegraphics[width=1\linewidth]{Code/Fig/ode_sin_input_5.65}
			\caption{Ode Plot for $\omega$ = 5.65}
			\label{fig:odesininput5_65}
		\end{figure}
		The magnitude of system response Figure \ref{fig:odesininput5_65} is $\approx$ 2,while the phase shift is $\approx$ -1.3 rad.
		%%% Insert Figure Here as a .pdf File %%%
		\begin{figure}[H]
			\centering
			\includegraphics[width=1\linewidth]{Code/Fig/ode_sin_input_16}
			\caption{Ode Plot for $\omega$ = 16}
			\label{fig:odesininput16}
		\end{figure}		
		The magnitude of system response of Figure \ref{fig:odesininput16} is $\approx$ 1.6, while the phase shift is $\approx$  -3.3 rad.
		
		\newpage
		\subsection{Bode Plot}
		%%% INSERT CODE HERE AS A .m FILE %%%
		\lstinputlisting[language=Matlab, caption = {\Large Bode Plot Code for State Space Function}]{Code/bode.m}
		
		\begin{figure}[H]
			\centering
			\includegraphics[width=1\linewidth]{Code/Fig/bode_plot}
			\caption{Bode Plot for State Space Model}
			\label{fig:bodeplot}
		\end{figure}
	
		\begin{itemize}
			\item At $\omega$ = 0.1, the amplitude is 1.000 and the phase is -0.008
			\item At $\omega$ = 1, the amplitude is 1.025 and the phase is -0.0855
			\item At $\omega$ = 5.65, the amplitude is 2.065 and the phase is -1.334
			\item At $\omega$ = 16, the amplitude is 0.160 and the phase is -2.927
		\end{itemize}
		
		The steady state values obtained from the simulations are near identical to the frequency free-response calculations. 
	\section{Conclusion}
	%%% PUT CONCLUSION HERE %%
	This MATLAB lab successfully explored the intricacies and challenges of numerically solving differential equations. By constructing a dynamic system simulation, the students gained practical experience in applying various numerical methods and interpreting the results.
	
	\begin{itemize}
		\item Step Size
			\begin{itemize}
				\item Smaller step sizes (h < 0.01) enhance accuracy and stabilize the simulation, but at a 10x computational cost increase per 10x step size reduction.
				\item Avoid large step sizes (h > 0.01) to ensure stable and reliable simulations.
			\end{itemize}
		\item Frequency
			\begin{itemize}
				\item Both Euler integration and sde23 methods accurately predict the expected settling time of $T_s = 3.066\overline{6}$, solidifying the validity of the simulation in a steady-state context.
				\item This method effectively predicts the magnitude and phase of the steady-state response, further strengthening confidence in the overall simulation's accuracy.
			\end{itemize}
			
	\end{itemize}
	

	
\end{document}